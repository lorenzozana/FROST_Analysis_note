
\tableofcontents
\section{Introduction}
In addition to requiring both proton and neutron targets, a full experimental understanding of the photoproduction reaction system requires measurements
beyond that of the unpolarised differential cross section. This is because the four CGLN structure functions arise from the four possible combinations
of photon helicity and nucleon spin. Experiments involving polarised beams and/or polarised nucleon targets are therefore required, with the CGLN structure
functions being most easily related to these experiments in terms of helicity or transversity amplitudes. \\
The 16 polarisation observables are classified as: the differential cross section, three single-polarisation observables ($P, \Sigma$, and $T$ ) where one of the beam, target
or recoil are polarised, and 12 double polarisation observables where two of the three reaction components which can carry polarisation are polarised. The
double-polarisation observables themselves are divided into three groups: beam-target ($G, H, E, F$ ), beam-recoil ($O_x , O_z, C_x , C_z$), and target-recoil ($T_x , L_x , L_z$).
Experimentally, it is the differential cross section of the meson in the photo-production reaction that will be measured. For the beam-target measurements, this can be expressed in terms of polarisation observables as [39]:
$$
\frac{d\sigma}{d\omega} = \left(\frac{d\sigma}{d\omega} \right)_{unpolarized}  \left{ 1 - P_L \Sigma cos(2\phi) + P_x \left[-P_L H sin(2\phi) + P_{\bigodot}F\right] -P_y \left[ -T +P_L P cos(2\phi)\right] -P_z \left[-P_L G sin(2\phi) + P_{\bigodot}E\right] \right} 
$$

\section{The g9 experiment}
The experimental Hall B at Jefferson Lab provided a unique set of experimental devices for the FROST experiment. The CEBAF Large Acceptance Spectrometer (CLAS)\cite{CLAS} , which was housed
in Hall B, was a nearly-4π spectrometer optimized for hadron spectroscopy. The bremsstrahlung tagging technique, which was used by the broad-range photon tagging facility Sober\cite{Sober_2000} at Hall B, could
tag photon energies over a range from 20\% to 95\% of the incident electron energy and was capable of operating with CEBAF beam energies up to 5.5 GeV. The remaining element which was
indispensable for the double-polarization experiments was the frozen-spin target FROST \cite{Keith_2012}. The FROST experiment used butanol as the ideal target material with a theoretical dilution factor of approximately 13.5\%. This material was dynamically polarized outside the CLAS spectrometer using a homogeneous magnetic field of about 5.0 T and cooled to approximately 0.5 K. Once polarized, the target was then cooled down to a low temperature of 30 mK, enough to preserve the nucleon polarization in a more moderate holding field of about 0.5 T. The target was then moved back into the CLAS spectrometer, and data acquisition with a tagged photon beam could commence (or
continue). The FROST experiment covered all possible combinations of beam and target polarizations. The experiment utilized a linearly- or circularly-polarized photon beam in combination with a longitudinally- (FROST-g9a) or transversely-polarized (FROST-g9b) target. The energy range covered in these experiments was up to 3.0 GeV in the runs with circularly-polarized photons and 2.1 GeV in the runs with linearly-polarized photons. In addition to the polarized butanol target, the experiment also used carbon and polyethylene targets which were kept further downstream in the target cryostat. They were useful for various systematics checks and for the determination of
the contribution of bound nucleons in the butanol data.

\section{Analysis procedure}
\subsection{Cuts}



TIPS:
You can get started by \textbf{double clicking} this text block and begin editing. You can also click the \textbf{Text} button below to add new block elements. Or you can \textbf{drag and drop an image} right onto this text. Happy writing!
