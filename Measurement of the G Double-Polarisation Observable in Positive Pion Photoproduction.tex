\documentclass{article}
\usepackage[affil-it]{authblk}
\usepackage{graphicx}
\usepackage[space]{grffile}
\usepackage{latexsym}
\usepackage{textcomp}
\usepackage{longtable}
\usepackage{multirow,booktabs}
\usepackage{amsfonts,amsmath,amssymb}
\usepackage{natbib}
\usepackage{url}
\usepackage{hyperref}
\hypersetup{colorlinks=false,pdfborder={0 0 0}}
% You can conditionalize code for latexml or normal latex using this.
\newif\iflatexml\latexmlfalse
\usepackage[utf8]{inputenc}
\usepackage[greek,english]{babel}




\begin{document}

\title{Measurement of the G Double-Polarisation Observable in Positive Pion Photoproduction}


\author{Lorenzo Zana}
\affil{The University of Edinburgh}

  
\author{Daniel watts}
\affil{Affiliation not available}
  



\date{\today}

\maketitle

\begin{abstract}
This analysis note will present the detailed measurement of positive pion photoproduction in the 730-2300 MeV photon energy (1400-2280 MeV centre-of-mass energy) region with a linearly polarised photon beam and a longitudinally polarised proton target with a close-to-complete angular coverage in detection of the reaction products. This unique set up allows for the extraction of the double-polarisation observable, G. The data were taken as part of the g9 experiment using a tagged, polarised photon beam and the Frozen Proton Spin Target (FROST)

%
\end{abstract}%




\tableofcontents

\section{The g9 experiment}
The experimental Hall B at Jefferson Lab provided a unique set of experimental devices for the
FROST experiment. The CEBAF Large Acceptance Spectrometer (CLAS)\cite{CLAS} , which was housed
in Hall B, was a nearly-4\selectlanguage{greek}π \selectlanguage{english}spectrometer optimized for hadron spectroscopy. The bremsstrahlung
tagging technique, which was used by the broad-range photon tagging facility Sober\cite{Sober_2000} at Hall B, could
tag photon energies over a range from 20\% to 95\% of the incident electron energy and was ca-
pable of operating with CEBAF beam energies up to 5.5 GeV. The remaining element which was
indispensable for the double-polarization experiments was the frozen-spin target FROST \cite{Keith_2012}. The
FROST experiment used butanol as the ideal target material with a theoretical dilution factor of
approximately 13.5\%. This material was dynamically polarized outside the CLAS spectrometer using a homogeneous magnetic field of about 5.0 T and cooled to approximately 0.5 K. Once polarized,
the target was then cooled down to a low temperature of 30 mK, enough to preserve the nucleon
polarization in a more moderate holding field of about 0.5 T. The target was then moved back
into the CLAS spectrometer, and data acquisition with a tagged photon beam could commence (or
continue). The FROST experiment covered all possible combinations of beam and target polarizations. The experiment utilized a linearly- or circularly-polarized photon beam in combination with
a longitudinally- (FROST-g9a) or transversely-polarized (FROST-g9b) target. The energy range
covered in these experiments was up to 3.0 GeV in the runs with circularly-polarized photons and
2.1 GeV in the runs with linearly-polarized photons. In addition to the polarized butanol target,
the experiment also used carbon and polyethylene targets which were kept further downstream in
the target cryostat. They were useful for various systematics checks and for the determination of
the contribution of bound nucleons in the butanol data.

\section{Analysis procedure}
\subsection{Cuts}



TIPS:
You can get started by \textbf{double clicking} this text block and begin editing. You can also click the \textbf{Text} button below to add new block elements. Or you can \textbf{drag and drop an image} right onto this text. Happy writing!

\bibliographystyle{plain}
\bibliography{bibliography/biblio.bib%
}

\end{document}

