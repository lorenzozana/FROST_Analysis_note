\section{Introduction}
In addition to requiring both proton and neutron targets, a full experimental understanding of the photoproduction reaction system requires measurements
beyond that of the unpolarised differential cross section. This is because the four CGLN structure functions arise from the four possible combinations
of photon helicity and nucleon spin. Experiments involving polarised beams and/or polarised nucleon targets are therefore required, with the CGLN structure
functions being most easily related to these experiments in terms of helicity or transversity amplitudes. \\
The 16 polarisation observables are classified as: the differential cross section, three single-polarisation observables ($P, \Sigma$, and $T$ ) where one of the beam, target
or recoil are polarised, and 12 double polarisation observables where two of the three reaction components which can carry polarisation are polarised. The
double-polarisation observables themselves are divided into three groups: beam-target ($G, H, E, F$ ), beam-recoil ($O_x , O_z, C_x , C_z$), and target-recoil ($T_x , L_x , L_z$).
Experimentally, it is the differential cross section of the meson in the photo-production reaction that will be measured. For the beam-target measurements, this can be expressed in terms of polarisation observables as \cite{Bark_1974}:
\begin{equation}
\frac{d\sigma}{d\omega} = \left(\frac{d\sigma}{d\omega} \right)_{unpol}  \left\{ 
\begin{aligned}
    & 1 - P_L \Sigma cos(2\phi) + P_x \left[-P_L H sin(2\phi) + P_{\bigodot}F\right] + \\
& -P_y \left[ -T +P_L P cos(2\phi)\right] -P_z \left[-P_L G sin(2\phi) + P_{\bigodot}E\right]
\end{aligned}
\right\} 
\end{equation}
