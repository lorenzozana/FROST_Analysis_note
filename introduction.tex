\section{Introduction}
The central physics motivation for measurements of meson photoproduction from nucleons is to provide data to constrain the excitation spectrum of the nucleon and the properties (spin, parity) of the resonant states.  This is possible as many $N^*$ states have a strong $N^*\rightarrow N + meson$  decay branch.  The meson photoproduction data are analyzed using partial wave analysis techniques to obtain sensitivity to the contributing $N^*$ states and background terms.
In addition to requiring both proton and neutron targets, a full experimental understanding of the photoproduction reaction system requires measurements
beyond that of the unpolarized differential cross section. Experiments involving polarized beams and/or polarized nucleon targets are therefore required. \\
The meson photoproduction process depends on four complex helicity amplitudes \cite{PhysRev.106.1337} \cite{PhysRev.106.1345}. These result in 16 independent polarization observables: the differential cross section, three single-polarization observables ($P, \Sigma$, and $T$ ) where one of the beam, target
or recoil are polarized, and 12 double polarization observables where two of the three reaction components which can carry polarization are polarized. The
double-polarization observables themselves are divided into three groups: beam-target ($G, H, E, F$ ), beam-recoil ($O_x , O_z, C_x , C_z$), and target-recoil ($T_x , L_x , L_z$).
The cross section for pseudo-scalar meson photoproduction from a nucleon target can be expressed as \cite{Bark_1974}:
\begin{equation}
\frac{d\sigma}{d\Omega} = \left(\frac{d\sigma}{d\Omega} \right)_{unpol}  \left\{ 
\begin{aligned}
    & 1 - P_L \Sigma cos(2\eta) + P_x \left[-P_L H sin(2\eta) + P_{\bigodot}F\right] + \\
& -P_y \left[ -T +P_L P cos(2\eta)\right] -P_z \left[-P_L G sin(2\eta) + P_{\bigodot}E\right],
\end{aligned}
\right\} 
\label{eqn:CGLN}
\end{equation}
where $P_L$ and $P_{\bigodot}$ is the degree of linear and circular polarisation, $P_x$, $P_y$, $P_z$ is the degree of target polarisation, and $\eta$ is the angle between the photon polarisation vector and the reaction plane.  
This analysis note describes the extraction of the G polarization observable, extracted from pion photo-production data obtained with a spin polarized proton (FROST) target. The CLAS detector in HallB at JLAB allowed the final state $\pi^+$ to be determined over a wide kinematic range. A data set covering $E_{Center\, of\, Mass}$ range from 1.4GeV to 2.29GeV with close to complete $\theta_{CM}$ coverage will provide valuable new constraints for partial wave analysis and fully determining the $\gamma (p,n)\pi^+$ final state
