\section{The g9 experiment}
The g9 experiment comprised 26 billion triggers produced by impinging  a  polarised photon beam on a polarized proton target placed at the centre of the CLAS spectrometer. The experiment has been described in detail elsewhere (see for example \cite{Strauch_2014}), so only a brief description will be given here. 
 The CEBAF Large Acceptance Spectrometer (CLAS)\cite{CLAS} was a nearly-4$\pi$ spectrometer optimized for hadron spectroscopy. The bremsstrahlung tagging technique, which was used by the broad-range photon tagging facility (Sober\cite{Sober_2000}) in Hall B, could tag photon energies over a range from 20\% to 95\% of the incident electron energy and was capable of operating with CEBAF beam energies up to 5.5 GeV. \\
The polarised protons were enabled by the Frozen spin target (FROST \cite{Keith_2012}) which used polarized butanol as the target material.
The butanol contained a high fraction of polarised protons and had a dilution (fraction of unpolarised nuclei) of only $\sim 13.5\%$ (see chapter \ref{ch:dil_factor} for a detailed evaluation and discussion).
 The butanol was dynamically polarized outside the CLAS spectrometer using a homogeneous magnetic field of about 5.0 T and cooled to approximately 0.5 K. Once polarized, the target was then cooled down to a low temperature of 30 mK, enough to preserve the nucleon polarization in a more moderate holding field of about 0.5 T (this was achieved using a solenoidal magnet coil housed around the target cell). The target was then moved back into the CLAS spectrometer, and data acquisition with a tagged photon beam could commence (or continue). The FROST experiment covered all possible combinations of beam and target polarizations. The experiment utilized a linearly- or circularly-polarized photon beam in combination with a longitudinally- (FROST-g9a) or transversely-polarized (FROST-g9b) target. The energy range covered in these experiments was up to 3.0 GeV in the runs with circularly-polarized photons and 2.3 GeV in the runs with linearly-polarized photons. In addition to the polarized butanol target, the experiment also used carbon and polyethylene targets which were kept further downstream in the target cryostat. They were used for various systematics checks and for the determination of the contribution of unpolarised bound nucleons in the carbon and oxygen nuclei of the butanol.
