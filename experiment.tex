\section{The g9 experiment}
The experimental Hall B at Jefferson Lab provided a unique set of experimental devices for the FROST experiment. The CEBAF Large Acceptance Spectrometer (CLAS)\cite{CLAS}, which was housed in Hall B, was a nearly-4$\pi$ spectrometer optimized for hadron spectroscopy. The bremsstrahlung tagging technique, which was used by the broad-range photon tagging facility Sober\cite{Sober_2000} at Hall B, could tag photon energies over a range from 20\% to 95\% of the incident electron energy and be capable of operating with CEBAF beam energies up to 5.5 GeV. The remaining element which was indispensable for the double-polarization experiments was the frozen-spin target FROST \cite{Keith_2012}. The FROST experiment used butanol as the ideal target material with a theoretical dilution factor of approximately 13.5\%. This material was dynamically polarized outside the CLAS spectrometer using a homogeneous magnetic field of about 5.0 T and cooled to approximately 0.5 K. Once polarized, the target was then cooled down to a low temperature of 30 mK, enough to preserve the nucleon polarization in a more moderate holding field of about 0.5 T. The target was then moved back into the CLAS spectrometer, and data acquisition with a tagged photon beam could commence (or continue). The FROST experiment covered all possible combinations of beam and target polarizations. The experiment utilized a linearly- or circularly-polarized photon beam in combination with a longitudinally- (FROST-g9a) or transversely-polarized (FROST-g9b) target. The energy range covered in these experiments was up to 3.0 GeV in the runs with circularly-polarized photons and 2.1 GeV in the runs with linearly-polarized photons. In addition to the polarized butanol target, the experiment also used carbon and polyethylene targets which were kept further downstream in the target cryostat. They were used for various systematics checks and for the determination of the contribution of bound nucleons in the butanol data.
