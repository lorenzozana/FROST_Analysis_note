\section{Appendix}

\subsection{G vs \texorpdfstring{$E_{CMS}$}{E-CMS} for different \texorpdfstring{$cos(\theta)$}{cos(theta)} bins and different Target polarization}
\label{app:G_TGPOL}
The value of G is extracted independently from different target polarizations (positive (GREEN)  and negative (BLUE) ). Results are compared in order to investigate possible systematic difference between the two sets. Here are plotted vs $E_{CMS}$(W) for different $cos(\theta_{CM})$ bins. \begin{figure}[htb]
  \begin{center}
    \subfloat[][ $ -1 \leq cos(\theta_{CM})<-0.8$ ] {
      \includegraphics[height=0.40\textwidth]{figures/G_plots/comp_TGPOL_system_0.png} 
      \label{fig:GvsW_TGPOL_theta1}
    }
    \subfloat[][ $ -0.8 \leq cos(\theta_{CM})<-0.6$ ] {
      \includegraphics[height=0.40\textwidth]{figures/G_plots/comp_TGPOL_system_1.png}
    \label{fig:GvsW_TGPOL_theta2}
    } \\
    \subfloat[][ $ -0.6 \leq cos(\theta_{CM})<-0.4$ ] {
    \includegraphics[height=0.40\textwidth]{figures/G_plots/comp_TGPOL_system_2.png}
    \label{fig:GvsW_TGPOL_theta3}
    }
    \subfloat[][ $ -0.4 \leq cos(\theta_{CM})<-0.2$ ] {
      \includegraphics[height=0.40\textwidth]{figures/G_plots/comp_TGPOL_system_3.png}
      \label{fig:GvsW_TGPOL_theta4}
    }
  \end{center}
\end{figure}
\begin{figure}[htb]
\ContinuedFloat
  \begin{center} 
    \subfloat[][ $ -0.2 \leq cos(\theta_{CM})<0.0$ ] {
      \includegraphics[height=0.40\textwidth]{figures/G_plots/comp_TGPOL_system_4.png}
      \label{fig:GvsW_TGPOL_theta5}
    }
    \subfloat[][ $ 0.0 \leq cos(\theta_{CM})<0.2$ ] {
      \includegraphics[height=0.40\textwidth]{figures/G_plots/comp_TGPOL_system_5.png}
      \label{fig:GvsW_TGPOL_theta6}
    } \\    
    \subfloat[][ $ 0.2 \leq cos(\theta_{CM})<0.4$ ] {
      \includegraphics[height=0.40\textwidth]{figures/G_plots/comp_TGPOL_system_6.png}
      \label{fig:GvsW_TGPOL_theta7}
    }
    \subfloat[][ $ 0.4 \leq cos(\theta_{CM})<0.6$ ] {
      \includegraphics[height=0.40\textwidth]{figures/G_plots/comp_TGPOL_system_7.png}
      \label{fig:GvsW_TGPOL_theta8}
    }
  \end{center}
\end{figure}
\begin{figure}[htb]
\ContinuedFloat
  \begin{center} 
    \subfloat[][ $ 0.6 \leq cos(\theta_{CM})<0.8$ ] {
      \includegraphics[height=0.40\textwidth]{figures/G_plots/comp_TGPOL_system_8.png}
      \label{fig:GvsW_TGPOL_theta9}
    }
    \subfloat[][ $ 0.8 \leq cos(\theta_{CM})<1.0$ ] {
      \includegraphics[height=0.40\textwidth]{figures/G_plots/comp_TGPOL_system_9.png}
      \label{fig:GvsW_TGPOL_theta10}
    }
  \end{center}
\end{figure}

\clearpage
\newpage

\subsection{\texorpdfstring{$\Sigma$}{Sigma} vs \texorpdfstring{$E_{CMS}$}{E-CMS} for different \texorpdfstring{$cos(\theta)$}{cos(theta)} bins and different Target polarization}
\label{app:Sigma_TGPOL}
The value of $\Sigma$ is extracted independently from different target polarizations (positive (GREEN)  and negative (BLUE) ). Results are compared in order to investigate possible systematic difference between the two sets. Here are plotted vs $E_{CMS}$(W) for different $cos(\theta_{CM})$ bins.
\begin{figure}[htb]
  \begin{center}
    \subfloat[][ $ -1 \leq cos(\theta_{CM})<-0.8$ ] {
      \includegraphics[height=0.40\textwidth]{figures/G_plots/comp_SIGMA_TGPOL_system_0.png} 
      \label{fig:GvsW_SIGMA_TGPOL_theta1}
    }
    \subfloat[][ $ -0.8 \leq cos(\theta_{CM})<-0.6$ ] {
      \includegraphics[height=0.40\textwidth]{figures/G_plots/comp_SIGMA_TGPOL_system_1.png}
    \label{fig:GvsW_SIGMA_TGPOL_theta2}
    } \\
    \subfloat[][ $ -0.6 \leq cos(\theta_{CM})<-0.4$ ] {
    \includegraphics[height=0.40\textwidth]{figures/G_plots/comp_SIGMA_TGPOL_system_2.png}
    \label{fig:GvsW_SIGMA_TGPOL_theta3}
    }
    \subfloat[][ $ -0.4 \leq cos(\theta_{CM})<-0.2$ ] {
      \includegraphics[height=0.40\textwidth]{figures/G_plots/comp_SIGMA_TGPOL_system_3.png}
      \label{fig:GvsW_SIGMA_TGPOL_theta4}
    }
  \end{center}
\end{figure}
\begin{figure}[htb]
\ContinuedFloat
  \begin{center} 
    \subfloat[][ $ -0.2 \leq cos(\theta_{CM})<0.0$ ] {
      \includegraphics[height=0.40\textwidth]{figures/G_plots/comp_SIGMA_TGPOL_system_4.png}
      \label{fig:GvsW_SIGMA_TGPOL_theta5}
    }
    \subfloat[][ $ 0.0 \leq cos(\theta_{CM})<0.2$ ] {
      \includegraphics[height=0.40\textwidth]{figures/G_plots/comp_SIGMA_TGPOL_system_5.png}
      \label{fig:GvsW_SIGMA_TGPOL_theta6}
    } \\    
    \subfloat[][ $ 0.2 \leq cos(\theta_{CM})<0.4$ ] {
      \includegraphics[height=0.40\textwidth]{figures/G_plots/comp_SIGMA_TGPOL_system_6.png}
      \label{fig:GvsW_SIGMA_TGPOL_theta7}
    }
    \subfloat[][ $ 0.4 \leq cos(\theta_{CM})<0.6$ ] {
      \includegraphics[height=0.40\textwidth]{figures/G_plots/comp_SIGMA_TGPOL_system_7.png}
      \label{fig:GvsW_SIGMA_TGPOL_theta8}
    }
  \end{center}
\end{figure}
\begin{figure}[htb]
\ContinuedFloat
  \begin{center} 
    \subfloat[][ $ 0.6 \leq cos(\theta_{CM})<0.8$ ] {
      \includegraphics[height=0.40\textwidth]{figures/G_plots/comp_SIGMA_TGPOL_system_8.png}
      \label{fig:GvsW_SIGMA_TGPOL_theta9}
    }
    \subfloat[][ $ 0.8 \leq cos(\theta_{CM})<1.0$ ] {
      \includegraphics[height=0.40\textwidth]{figures/G_plots/comp_SIGMA_TGPOL_system_9.png}
      \label{fig:GvsW_SIGMA_TGPOL_theta10}
    }
  \end{center}
\end{figure}


\clearpage
\newpage

\subsection{\texorpdfstring{$\gamma$}{gamma} Polarization vs \texorpdfstring{$E_{CMS}$}{E-CMS} for different \texorpdfstring{$cos(\theta)$}{cos(theta} bins and different Target polarization}
 Photon polarization for the different bins in $E_{CMS}$ and $cos(\theta_{CM})$ where we extracted G and $\Sigma$ are plotted for different Target polarizations (positive (GREEN) and negative (BLUE)). The first point for negative Polarization in bin 1 showed a Polarization below the threshold of 50\% and wasn't used in the extraction (This particular set was missing part of the photon polarization information, inducing the use of integrated Polarization values (and not weighted $\phi$ distributions)).
\begin{figure}[htb]
  \begin{center}
    \subfloat[][ $ -1 \leq cos(\theta_{CM})<-0.8$ ] {
      \includegraphics[height=0.40\textwidth]{figures/G_plots/comp_GP_TGPOL_system_0.png} 
      \label{fig:GvsW_GP_TGPOL_theta1}
    }
    \subfloat[][ $ -0.8 \leq cos(\theta_{CM})<-0.6$ ] {
      \includegraphics[height=0.40\textwidth]{figures/G_plots/comp_GP_TGPOL_system_1.png}
    \label{fig:GvsW_GP_TGPOL_theta2}
    } \\
    \subfloat[][ $ -0.6 \leq cos(\theta_{CM})<-0.4$ ] {
    \includegraphics[height=0.40\textwidth]{figures/G_plots/comp_GP_TGPOL_system_2.png}
    \label{fig:GvsW_GP_TGPOL_theta3}
    }
    \subfloat[][ $ -0.4 \leq cos(\theta_{CM})<-0.2$ ] {
      \includegraphics[height=0.40\textwidth]{figures/G_plots/comp_GP_TGPOL_system_3.png}
      \label{fig:GvsW_GP_TGPOL_theta4}
    }
  \end{center}
\end{figure}
\begin{figure}[htb]
\ContinuedFloat
  \begin{center} 
    \subfloat[][ $ -0.2 \leq cos(\theta_{CM})<0.0$ ] {
      \includegraphics[height=0.40\textwidth]{figures/G_plots/comp_GP_TGPOL_system_4.png}
      \label{fig:GvsW_GP_TGPOL_theta5}
    }
    \subfloat[][ $ 0.0 \leq cos(\theta_{CM})<0.2$ ] {
      \includegraphics[height=0.40\textwidth]{figures/G_plots/comp_GP_TGPOL_system_5.png}
      \label{fig:GvsW_GP_TGPOL_theta6}
    } \\    
    \subfloat[][ $ 0.2 \leq cos(\theta_{CM})<0.4$ ] {
      \includegraphics[height=0.40\textwidth]{figures/G_plots/comp_GP_TGPOL_system_6.png}
      \label{fig:GvsW_GP_TGPOL_theta7}
    }
    \subfloat[][ $ 0.4 \leq cos(\theta_{CM})<0.6$ ] {
      \includegraphics[height=0.40\textwidth]{figures/G_plots/comp_GP_TGPOL_system_7.png}
      \label{fig:GvsW_GP_TGPOL_theta8}
    }
  \end{center}
\end{figure}
\begin{figure}[htb]
\ContinuedFloat
  \begin{center} 
    \subfloat[][ $ 0.6 \leq cos(\theta_{CM})<0.8$ ] {
      \includegraphics[height=0.40\textwidth]{figures/G_plots/comp_GP_TGPOL_system_8.png}
      \label{fig:GvsW_GP_TGPOL_theta9}
    }
    \subfloat[][ $ 0.8 \leq cos(\theta_{CM})<1.0$ ] {
      \includegraphics[height=0.40\textwidth]{figures/G_plots/comp_GP_TGPOL_system_9.png}
      \label{fig:GvsW_GP_TGPOL_theta10}
    }
  \end{center}
\end{figure}


\clearpage
\newpage

\subsection{Dilution Factor vs \texorpdfstring{$E_{CMS}$}{E-CMS} for different \texorpdfstring{$cos(\theta_{CM})$}{cos(theta-CM)} bins and different Target polarization}
\label{app:dilfactor}
The Dilution Factor is plotted  for the different bins in $E_{CMS}$ and $cos(\theta_{CM})$ where we extracted G and $\Sigma$. A full explanation of the extraction method and the plots is shown in chapter \ref{ch:dil_factor}. Highlighted in different colors are shown the 3 different electron beam energies used in this data-set. Different beam energies corresponded to different photon resolution and we expected a difference in the calculated dilution factor. For these reason, the 3 settings were threated differently. 
\begin{figure}[htb]
  \begin{center}
    \subfloat[][ $ -1 \leq cos(\theta_{CM})<-0.8$ ] {
      \includegraphics[height=0.40\textwidth]{figures/G_plots/comp_DF_TGPOL_system_0.png} 
      \label{fig:GvsW_DF_TGPOL_theta1}
    }
    \subfloat[][ $ -0.8 \leq cos(\theta_{CM})<-0.6$ ] {
      \includegraphics[height=0.40\textwidth]{figures/G_plots/comp_DF_TGPOL_system_1.png}
    \label{fig:GvsW_DF_TGPOL_theta2}
    } \\
    \subfloat[][ $ -0.6 \leq cos(\theta_{CM})<-0.4$ ] {
    \includegraphics[height=0.40\textwidth]{figures/G_plots/comp_DF_TGPOL_system_2.png}
    \label{fig:GvsW_DF_TGPOL_theta3}
    }
    \subfloat[][ $ -0.4 \leq cos(\theta_{CM})<-0.2$ ] {
      \includegraphics[height=0.40\textwidth]{figures/G_plots/comp_DF_TGPOL_system_3.png}
      \label{fig:GvsW_DF_TGPOL_theta4}
    }
  \end{center}
\end{figure}
\begin{figure}[htb]
\ContinuedFloat
  \begin{center} 
    \subfloat[][ $ -0.2 \leq cos(\theta_{CM})<0.0$ ] {
      \includegraphics[height=0.40\textwidth]{figures/G_plots/comp_DF_TGPOL_system_4.png}
      \label{fig:GvsW_DF_TGPOL_theta5}
    }
    \subfloat[][ $ 0.0 \leq cos(\theta_{CM})<0.2$ ] {
      \includegraphics[height=0.40\textwidth]{figures/G_plots/comp_DF_TGPOL_system_5.png}
      \label{fig:GvsW_DF_TGPOL_theta6}
    } \\    
    \subfloat[][ $ 0.2 \leq cos(\theta_{CM})<0.4$ ] {
      \includegraphics[height=0.40\textwidth]{figures/G_plots/comp_DF_TGPOL_system_6.png}
      \label{fig:GvsW_DF_TGPOL_theta7}
    }
    \subfloat[][ $ 0.4 \leq cos(\theta_{CM})<0.6$ ] {
      \includegraphics[height=0.40\textwidth]{figures/G_plots/comp_DF_TGPOL_system_7.png}
      \label{fig:GvsW_DF_TGPOL_theta8}
    }
  \end{center}
\end{figure}
\begin{figure}[htb]
\ContinuedFloat
  \begin{center} 
    \subfloat[][ $ 0.6 \leq cos(\theta_{CM})<0.8$ ] {
      \includegraphics[height=0.40\textwidth]{figures/G_plots/comp_DF_TGPOL_system_8.png}
      \label{fig:GvsW_DF_TGPOL_theta9}
    }
    \subfloat[][ $ 0.8 \leq cos(\theta_{CM})<1.0$ ] {
      \includegraphics[height=0.40\textwidth]{figures/G_plots/comp_DF_TGPOL_system_9.png}
      \label{fig:GvsW_DF_TGPOL_theta10}
    }
  \end{center}
\end{figure}

\clearpage
\newpage

\subsection{Target polarization vs \texorpdfstring{$E_{CMS}$}{E-CMS} for different \texorpdfstring{$cos(\theta)$}{cos(theta)} bins and different Target polarization}
\label{app:tgpol}
Target polarization is plotted vs $E_{CMS}$(W) for different $cos(\theta_{CM})$ bins. Different Target polarizations have been given different colors : positive (GREEN), negative (BLUE)
\begin{figure}[htb]
  \begin{center}
    \subfloat[][ $ -1 \leq cos(\theta_{CM})<-0.8$ ] {
      \includegraphics[height=0.40\textwidth]{figures/G_plots/comp_TG_TGPOL_system_0.png} 
      \label{fig:GvsW_TG_TGPOL_theta1}
    }
    \subfloat[][ $ -0.8 \leq cos(\theta_{CM})<-0.6$ ] {
      \includegraphics[height=0.40\textwidth]{figures/G_plots/comp_TG_TGPOL_system_1.png}
    \label{fig:GvsW_TG_TGPOL_theta2}
    } \\
    \subfloat[][ $ -0.6 \leq cos(\theta_{CM})<-0.4$ ] {
    \includegraphics[height=0.40\textwidth]{figures/G_plots/comp_TG_TGPOL_system_2.png}
    \label{fig:GvsW_TG_TGPOL_theta3}
    }
    \subfloat[][ $ -0.4 \leq cos(\theta_{CM})<-0.2$ ] {
      \includegraphics[height=0.40\textwidth]{figures/G_plots/comp_TG_TGPOL_system_3.png}
      \label{fig:GvsW_TG_TGPOL_theta4}
    }
  \end{center}
\end{figure}
\begin{figure}[htb]
\ContinuedFloat
  \begin{center} 
    \subfloat[][ $ -0.2 \leq cos(\theta_{CM})<0.0$ ] {
      \includegraphics[height=0.40\textwidth]{figures/G_plots/comp_TG_TGPOL_system_4.png}
      \label{fig:GvsW_TG_TGPOL_theta5}
    }
    \subfloat[][ $ 0.0 \leq cos(\theta_{CM})<0.2$ ] {
      \includegraphics[height=0.40\textwidth]{figures/G_plots/comp_TG_TGPOL_system_5.png}
      \label{fig:GvsW_TG_TGPOL_theta6}
    } \\    
    \subfloat[][ $ 0.2 \leq cos(\theta_{CM})<0.4$ ] {
      \includegraphics[height=0.40\textwidth]{figures/G_plots/comp_TG_TGPOL_system_6.png}
      \label{fig:GvsW_TG_TGPOL_theta7}
    }
    \subfloat[][ $ 0.4 \leq cos(\theta_{CM})<0.6$ ] {
      \includegraphics[height=0.40\textwidth]{figures/G_plots/comp_TG_TGPOL_system_7.png}
      \label{fig:GvsW_TG_TGPOL_theta8}
    }
  \end{center}
\end{figure}
\begin{figure}[htb]
\ContinuedFloat
  \begin{center} 
    \subfloat[][ $ 0.6 \leq cos(\theta_{CM})<0.8$ ] {
      \includegraphics[height=0.40\textwidth]{figures/G_plots/comp_TG_TGPOL_system_8.png}
      \label{fig:GvsW_TG_TGPOL_theta9}
    }
    \subfloat[][ $ 0.8 \leq cos(\theta_{CM})<1.0$ ] {
      \includegraphics[height=0.40\textwidth]{figures/G_plots/comp_TG_TGPOL_system_9.png}
      \label{fig:GvsW_TG_TGPOL_theta10}
    }
  \end{center}
\end{figure}

\FloatBarrier
\subsection{Simulation Code}
\label{app:simcode}
Simulation code used in order to test the extraction of the variables (See chapted \ref{ch:sys_corr}). After different effects were evaluated, a faster version was used in order to address the systematic shifts induced by the method of extraction of the variables. This was needed in order to speed up the simulation time, since were used 100 simulations for each of the 28 different statistics, the 11 different G (letting it vary by 0.2 between -1 and 1) and 11 different $\Sigma$ investigated (a total 338800 simulations).
\begin{lstlisting}[language=C++]
#include "TH1F.h"
#include "TRandom2.h"
#include "TFile.h"
#include "TMath.h"
#include "TF1.h"

double low_tgpol = 0.78;
double high_tgpol = 0.85;

int target_pol = 1; // define target polarization

double mean_beampol = 0.6;
double sigma_beampol = 0.1;
double beampol_err = 0.10; // 10% error for polarization

double G_v = 0.5;
double Sigma_v = 0.3;

int tot_events = 300;

TRandom *fRandom;

double GetTG_pol(int event,int pol) {

  double tg_pol = double(pol) * fRandom->Uniform(low_tgpol,high_tgpol);
  
  return tg_pol;
}


double GetBeam_pol(int event,int pol) {
  double beam_pol;
  double at_mean_pol = mean_beampol + double(pol)*0.05;
  if ( fRandom->Rndm() < 0.5 ) {
    sigma_beampol = 0.1;
    beam_pol = fRandom->Gaus(0.0, sigma_beampol);
    beam_pol = double(pol) * (at_mean_pol + TMath::Abs(beam_pol));
  }
  else {
    sigma_beampol = 0.025;
    beam_pol = fRandom->Gaus(0.0, sigma_beampol);
    beam_pol =double(pol) *( at_mean_pol - TMath::Abs(beam_pol) );
 
  }
  return beam_pol;
}

Double_t dsigma_domega_pol(Double_t *x, Double_t *par)
{
  Float_t xx =x[0];
  Double_t f = 0.0;
  if(((-180<xx/TMath::Pi()*180.) && (xx/TMath::Pi()*180.<-155))||((-145<xx/TMath::Pi()*180.)&&(xx/TMath::Pi()*180.<-95))||((-85<xx/TMath::Pi()*180.)&&(xx/TMath::Pi()*180.<-35))||((-25<xx/TMath::Pi()*180.)&&(xx/TMath::Pi()*180.<25))||((35<xx/TMath::Pi()*180.)&&(xx/TMath::Pi()*180.<85))||((95<xx/TMath::Pi()*180.)&&(xx/TMath::Pi()*180.<145))||((155<xx/TMath::Pi()*180.)&&(xx/TMath::Pi()*180.<180))){
    f = par[0]*(1+par[1]*par[2] * cos(2*xx) + par[3] * par[2] *par[4] * sin(2*xx));  
    // par[0] = cross section unpolarized
    // par[1] = Sigma
    // par[2] = Beam Polarization
    // par[3] = G
    // par[4] = Target Polarization
  }
   return f;
}


Double_t Get_flat_amo() {

  Double_t xx = 0.0;
  Double_t yy ;
  while (xx == 0.0) {
    yy = fRandom->Uniform(-TMath::Pi(),TMath::Pi());
    if(((-180<yy/TMath::Pi()*180.) && (yy/TMath::Pi()*180.<-155))||((-145<yy/TMath::Pi()*180.)&&(yy/TMath::Pi()*180.<-95))||((-85<yy/TMath::Pi()*180.)&&(yy/TMath::Pi()*180.<-35))||((-25<yy/TMath::Pi()*180.)&&(yy/TMath::Pi()*180.<25))||((35<yy/TMath::Pi()*180.)&&(yy/TMath::Pi()*180.<85))||((95<yy/TMath::Pi()*180.)&&(yy/TMath::Pi()*180.<145))||((155<yy/TMath::Pi()*180.)&&(yy/TMath::Pi()*180.<180))) xx = yy; 
  }
  return xx;

}

void Fill_histo(TH1F *histo , float  val, float pol) {
  int at_bin = histo->FindBin(val);
  float  at_val = histo->GetBinContent(at_bin);
  float at_err = histo->GetBinError(at_bin);
  float pol_err = TMath::Abs(pol) * beampol_err;
  histo->Fill(val,TMath::Abs(1./pol));
  float val_err = at_err;
  if (pol != 0.0 && pol_err != 0.0 ) val_err = pow(1./pow(pol,2) + pow(pol_err,2)/pow(pol,4),0.5);
  histo->SetBinError(at_bin,pow(pow(at_err,2)+pow(val_err,2),0.5));

}

void Run(){
  fRandom = new TRandom2(0);
  double beam_pol, tg_pol;
  TFile *file_out = new TFile("output_test_polasym5_1000.root","RECREATE");

  TF1 *para_func = new TF1("para_func",dsigma_domega_pol,-TMath::Pi(),TMath::Pi(),5);
  TF1 *perp_func = new TF1("para_func",dsigma_domega_pol,-TMath::Pi(),TMath::Pi(),5);

  TH1F *para_h = new TH1F("para_h","PARA histogram; #phi",100,-TMath::Pi(),TMath::Pi());
  TH1F *perp_h = new TH1F("perp_h","PERP histogram; #phi",100,-TMath::Pi(),TMath::Pi());

  TH1F *para2_h = new TH1F("para2_h","PARA histogram weighted pol; #phi",100,-TMath::Pi(),TMath::Pi());
  TH1F *perp2_h = new TH1F("perp2_h","PERP histogram weighted pol; #phi",100,-TMath::Pi(),TMath::Pi());

  TH1F *amo_h = new TH1F("amo_h","AMO histogram; #phi",100,-TMath::Pi(),TMath::Pi());

  TH1F *beampol_0h = new TH1F("beampol_0h","Beam Polarization -1 ; Pol",100,0.,1.0);
  TH1F *beampol_1h = new TH1F("beampol_1h","Beam Polarization 1; Pol",100,0.,1.0);

  double val_phi;
  double meas_beampol;

  for (int j=0; j<2; j++) { // beam polarization
    for (int i =0 ; i< tot_events; i++) {
      if (i % 100 == 0) printf("Polarization %d at event %d \n",(j*2-1),i);

      if (i<int(tot_events * 0.8) ) {
	amo_h->Fill(Get_flat_amo());
      }
      beam_pol = GetBeam_pol(i,(j*2)-1);
      meas_beampol = fRandom->Gaus(beam_pol,sigma_beampol);
      if (j==0) beampol_0h->Fill(TMath::Abs(meas_beampol));
      else beampol_1h->Fill(TMath::Abs(meas_beampol));
   
      //      printf("Got Beam Pol \n");
      tg_pol = GetTG_pol(i,target_pol);
      //     printf("Got TG pol \n");
      if (j==0) {
	para_func->SetParameters(10.,Sigma_v,beam_pol,G_v,tg_pol);
	val_phi = para_func->GetRandom();
	para_h->Fill(val_phi);
	Fill_histo(para2_h,val_phi,meas_beampol);
      }

      if (j==1) {
	perp_func->SetParameters(10.,Sigma_v,beam_pol,G_v,tg_pol);
	val_phi = perp_func->GetRandom();
	perp_h->Fill(val_phi);
	Fill_histo(perp2_h,val_phi,meas_beampol);
      }

    }
    
  }
  para_h->Write();
  perp_h->Write();
  para2_h->Write();
  perp2_h->Write();
  amo_h->Write();
  beampol_0h->Write();
  beampol_1h->Write();

  file_out->Close();


}


void Run_grid(int events,int G_bin,int Sigma_bin){
  fRandom = new TRandom2(0);
  double beam_pol, tg_pol;
  char name_out[30];
  sprintf(name_out,"output_test_Ev%d_G%d_S%d.root",events,G_bin,Sigma_bin);
  TFile *file_out = new TFile(name_out,"RECREATE");
  double G_at = -1.0 + double(G_bin)*0.2;
  double Sigma_at = -1.0 + double(Sigma_bin)*0.2;

  TF1 *para_func = new TF1("para_func",dsigma_domega_pol,-TMath::Pi(),TMath::Pi(),5);
  TF1 *perp_func = new TF1("para_func",dsigma_domega_pol,-TMath::Pi(),TMath::Pi(),5);

  TH1F *para_h = new TH1F("para_h","PARA histogram; #phi",100,-TMath::Pi(),TMath::Pi());
  TH1F *perp_h = new TH1F("perp_h","PERP histogram; #phi",100,-TMath::Pi(),TMath::Pi());

  TH1F *para2_h = new TH1F("para2_h","PARA histogram weighted pol; #phi",100,-TMath::Pi(),TMath::Pi());
  TH1F *perp2_h = new TH1F("perp2_h","PERP histogram weighted pol; #phi",100,-TMath::Pi(),TMath::Pi());

  TH1F *amo_h = new TH1F("amo_h","AMO histogram; #phi",100,-TMath::Pi(),TMath::Pi());

  TH1F *beampol_0h = new TH1F("beampol_0h","Beam Polarization -1 ; Pol",100,0.,1.0);
  TH1F *beampol_1h = new TH1F("beampol_1h","Beam Polarization 1; Pol",100,0.,1.0);

  double val_phi;
  double meas_beampol;

  for (int j=0; j<2; j++) { // beam polarization
    for (int i =0 ; i< events; i++) {
      if (i % 100 == 0) printf("Polarization %d at event %d \n",(j*2-1),i);

      if (i<int(tot_events * 0.8) ) {
	amo_h->Fill(Get_flat_amo());
      }
      beam_pol = GetBeam_pol(i,(j*2)-1);
      meas_beampol = fRandom->Gaus(beam_pol,sigma_beampol);
      if (j==0) beampol_0h->Fill(TMath::Abs(meas_beampol));
      else beampol_1h->Fill(TMath::Abs(meas_beampol));
   
      //      printf("Got Beam Pol \n");
      tg_pol = GetTG_pol(i,target_pol);
      //     printf("Got TG pol \n");
      if (j==0) {
	para_func->SetParameters(10.,Sigma_at,beam_pol,G_at,tg_pol);
	val_phi = para_func->GetRandom();
	para_h->Fill(val_phi);
	Fill_histo(para2_h,val_phi,meas_beampol);
      }

      if (j==1) {
	perp_func->SetParameters(10.,Sigma_at,beam_pol,G_at,tg_pol);
	val_phi = perp_func->GetRandom();
	perp_h->Fill(val_phi);
	Fill_histo(perp2_h,val_phi,meas_beampol);
      }

    }
    
  }
  para_h->Write();
  perp_h->Write();
  para2_h->Write();
  perp2_h->Write();
  amo_h->Write();
  beampol_0h->Write();
  beampol_1h->Write();

  file_out->Close();


}



void Run_total(){

  //  int run_events[5] = {300,500,700,1000,3000};
  // int run_events[4] = {500,700,1000,3000};
  int run_events[1] = {3000};

  for (int i=0; i<1; i++) { //events loop 
    for (int j=0; j<11 ; j++) { // G loop
      for (int k=1; k<10 ; k++) { //Sigma loop
	Run_grid(run_events[i],j,k);
      }
    }
  }

}


  
\end{lstlisting}

\subsection{Systematic Correction for G as a function of the estracted values of G and \texorpdfstring{$\Sigma$}{Sigma}}
\label{app:G_syscorr}
Systematic corrections to the extracted values of G are plotted vs different extracted value of G and $\Sigma$ for distinct statistics. The z axis scale is different at each plot. 
\begin{figure}[htb]
  \begin{center} 
    \subfloat[][$G^{sys}_{corr}$ vs ($G_{calc}$ and $\Sigma_{calc}$) \\ for $N_{events}=50$ ] {
      \includegraphics[width=0.33\textwidth]{figures/Sys_corr/Graph_G_50.pdf}
      \label{fig:Gcorr_sys_50}
    }
    \subfloat[][$G^{sys}_{corr}$ vs ($G_{calc}$ and $\Sigma_{calc}$) \\ for $N_{events}=60$ ] {
      \includegraphics[width=0.33\textwidth]{figures/Sys_corr/Graph_G_60.pdf}
      \label{fig:Gcorr_sys_60}
    }     
    \subfloat[][$G^{sys}_{corr}$ vs ($G_{calc}$ and $\Sigma_{calc}$) \\ for $N_{events}=70$ ] {
      \includegraphics[width=0.33\textwidth]{figures/Sys_corr/Graph_G_70.pdf}
      \label{fig:Gcorr_sys_70}
    }\\

    \subfloat[][$G^{sys}_{corr}$ vs ($G_{calc}$ and $\Sigma_{calc}$) \\ for $N_{events}=80$  ] {
      \includegraphics[width=0.33\textwidth]{figures/Sys_corr/Graph_G_80.pdf}
      \label{fig:Gcorr_sys_80}
    }
    \subfloat[][ $G^{sys}_{corr}$ vs ($G_{calc}$ and $\Sigma_{calc}$) \\ for $N_{events}=90$ ] {
      \includegraphics[width=0.33\textwidth]{figures/Sys_corr/Graph_G_90.pdf}
      \label{fig:Gcorr_sys_90}
    }
    \subfloat[][ $G^{sys}_{corr}$ vs ($G_{calc}$ and $\Sigma_{calc}$) \\ for $N_{events}=100$ ] {
      \includegraphics[width=0.33\textwidth]{figures/Sys_corr/Graph_G_100.pdf}
      \label{fig:Gcorr_sys_100}
    }
  \end{center}
\end{figure}
\begin{figure}[htb]
\ContinuedFloat
\begin{center}
    \subfloat[][$G^{sys}_{corr}$ vs ($G_{calc}$ and $\Sigma_{calc}$) \\ for $N_{events}=110$ ] {
      \includegraphics[width=0.33\textwidth]{figures/Sys_corr/Graph_G_110.pdf}
      \label{fig:Gcorr_sys_110}
    }
    \subfloat[][$G^{sys}_{corr}$ vs ($G_{calc}$ and $\Sigma_{calc}$) \\ for $N_{events}=120$ ] {
      \includegraphics[width=0.33\textwidth]{figures/Sys_corr/Graph_G_120.pdf}
      \label{fig:Gcorr_sys_120}
    }     
    \subfloat[][$G^{sys}_{corr}$ vs ($G_{calc}$ and $\Sigma_{calc}$) \\ for $N_{events}=130$ ] {
      \includegraphics[width=0.33\textwidth]{figures/Sys_corr/Graph_G_130.pdf}
      \label{fig:Gcorr_sys_130}
    } \\
    \subfloat[][$G^{sys}_{corr}$ vs ($G_{calc}$ and $\Sigma_{calc}$) \\ for $N_{events}=140$ ] {
      \includegraphics[width=0.33\textwidth]{figures/Sys_corr/Graph_G_140.pdf}
      \label{fig:Gcorr_sys_140}
    }
    \subfloat[][$G^{sys}_{corr}$ vs ($G_{calc}$ and $\Sigma_{calc}$) \\ for $N_{events}=150$ ] {
      \includegraphics[width=0.33\textwidth]{figures/Sys_corr/Graph_G_150.pdf}
      \label{fig:Gcorr_sys_150}
    }     
    \subfloat[][$G^{sys}_{corr}$ vs ($G_{calc}$ and $\Sigma_{calc}$) \\ for $N_{events}=160$ ] {
      \includegraphics[width=0.33\textwidth]{figures/Sys_corr/Graph_G_160.pdf}
      \label{fig:Gcorr_sys_160}
    }\\

    \subfloat[][$G^{sys}_{corr}$ vs ($G_{calc}$ and $\Sigma_{calc}$) \\ for $N_{events}=170$  ] {
      \includegraphics[width=0.33\textwidth]{figures/Sys_corr/Graph_G_170.pdf}
      \label{fig:Gcorr_sys_170}
    }
    \subfloat[][ $G^{sys}_{corr}$ vs ($G_{calc}$ and $\Sigma_{calc}$) \\ for $N_{events}=180$ ] {
      \includegraphics[width=0.33\textwidth]{figures/Sys_corr/Graph_G_180.pdf}
      \label{fig:Gcorr_sys_180}
    }
    \subfloat[][ $G^{sys}_{corr}$ vs ($G_{calc}$ and $\Sigma_{calc}$) \\ for $N_{events}=190$ ] {
      \includegraphics[width=0.33\textwidth]{figures/Sys_corr/Graph_G_190.pdf}
      \label{fig:Gcorr_sys_190}
    } \\
    \subfloat[][$G^{sys}_{corr}$ vs ($G_{calc}$ and $\Sigma_{calc}$) \\ for $N_{events}=200$ ] {
      \includegraphics[width=0.33\textwidth]{figures/Sys_corr/Graph_G_200.pdf}
      \label{fig:Gcorr_sys_200}
    }
    \subfloat[][$G^{sys}_{corr}$ vs ($G_{calc}$ and $\Sigma_{calc}$) \\ for $N_{events}=250$ ] {
      \includegraphics[width=0.33\textwidth]{figures/Sys_corr/Graph_G_250.pdf}
      \label{fig:Gcorr_sys_250}
    }     
    \subfloat[][$G^{sys}_{corr}$ vs ($G_{calc}$ and $\Sigma_{calc}$) \\ for $N_{events}=300$ ] {
      \includegraphics[width=0.33\textwidth]{figures/Sys_corr/Graph_G_300.pdf}
      \label{fig:Gcorr_sys_300}
    }\\
    \subfloat[][$G^{sys}_{corr}$ vs ($G_{calc}$ and $\Sigma_{calc}$) \\ for $N_{events}=350$ ] {
      \includegraphics[width=0.33\textwidth]{figures/Sys_corr/Graph_G_350.pdf}
      \label{fig:Gcorr_sys_350}
    }
    \subfloat[][$G^{sys}_{corr}$ vs ($G_{calc}$ and $\Sigma_{calc}$) \\ for $N_{events}=400$ ] {
      \includegraphics[width=0.33\textwidth]{figures/Sys_corr/Graph_G_400.pdf}
      \label{fig:Gcorr_sys_400}
    }     
    \subfloat[][$G^{sys}_{corr}$ vs ($G_{calc}$ and $\Sigma_{calc}$) \\ for $N_{events}=500$ ] {
      \includegraphics[width=0.33\textwidth]{figures/Sys_corr/Graph_G_500.pdf}
      \label{fig:Gcorr_sys_500}
    }\\
    \subfloat[][$G^{sys}_{corr}$ vs ($G_{calc}$ and $\Sigma_{calc}$) \\ for $N_{events}=600$ ] {
      \includegraphics[width=0.33\textwidth]{figures/Sys_corr/Graph_G_600.pdf}
      \label{fig:Gcorr_sys_600}
    }
    \subfloat[][$G^{sys}_{corr}$ vs ($G_{calc}$ and $\Sigma_{calc}$) \\ for $N_{events}=700$ ] {
      \includegraphics[width=0.33\textwidth]{figures/Sys_corr/Graph_G_700.pdf}
      \label{fig:Gcorr_sys_700}
    }     
    \subfloat[][$G^{sys}_{corr}$ vs ($G_{calc}$ and $\Sigma_{calc}$) \\ for $N_{events}=800$ ] {
      \includegraphics[width=0.33\textwidth]{figures/Sys_corr/Graph_G_800.pdf}
      \label{fig:Gcorr_sys_800}
    }
  \end{center}
\end{figure}
\begin{figure}[htb]
  \ContinuedFloat
  \begin{center}
    \subfloat[][$G^{sys}_{corr}$ vs ($G_{calc}$ and $\Sigma_{calc}$) \\ for $N_{events}=900$ ] {
      \includegraphics[width=0.33\textwidth]{figures/Sys_corr/Graph_G_200.pdf}
      \label{fig:Gcorr_sys_900}
    }
    \subfloat[][$G^{sys}_{corr}$ vs ($G_{calc}$ and $\Sigma_{calc}$) \\ for $N_{events}=1000$ ] {
      \includegraphics[width=0.33\textwidth]{figures/Sys_corr/Graph_G_1000.pdf}
      \label{fig:Gcorr_sys_1000}
    }     
    \subfloat[][$G^{sys}_{corr}$ vs ($G_{calc}$ and $\Sigma_{calc}$) \\ for $N_{events}=2000$ ] {
      \includegraphics[width=0.33\textwidth]{figures/Sys_corr/Graph_G_2000.pdf}
      \label{fig:Gcorr_sys_2000}
    } \\
    \subfloat[][$G^{sys}_{corr}$ vs ($G_{calc}$ and $\Sigma_{calc}$) \\ for $N_{events}=3000$ ] {
      \includegraphics[width=0.33\textwidth]{figures/Sys_corr/Graph_G_3000.pdf}
      \label{fig:Gcorr_sys_3000}
    }
  \end{center}
\end{figure}

\FloatBarrier
\newpage
\subsection{Systematic Correction for \texorpdfstring{$\Sigma$}{Sigma} as a function of the estracted values of G and \texorpdfstring{$\Sigma$}{Sigma}}
\label{app:Sigma_syscorr}
Systematic corrections to the extracted values of $\Sigma$ are plotted vs distinct extracted value of G and $\Sigma$ for different statistics. The z axis scale is different at each plot.
\begin{figure}[htb]
  \begin{center} 
    \subfloat[][$\Sigma^{sys}_{corr}$ vs ($G_{calc}$ and $\Sigma_{calc}$) \\ for $N_{events}=50$ ] {
      \includegraphics[width=0.33\textwidth]{figures/Sys_corr/Graph_Sigma_50.pdf}
      \label{fig:Sigmacorr_sys_50}
    }
    \subfloat[][$\Sigma^{sys}_{corr}$ vs ($G_{calc}$ and $\Sigma_{calc}$) \\ for $N_{events}=60$ ] {
      \includegraphics[width=0.33\textwidth]{figures/Sys_corr/Graph_Sigma_60.pdf}
      \label{fig:Sigmacorr_sys_60}
    }     
    \subfloat[][$\Sigma^{sys}_{corr}$ vs ($G_{calc}$ and $\Sigma_{calc}$) \\ for $N_{events}=70$ ] {
      \includegraphics[width=0.33\textwidth]{figures/Sys_corr/Graph_Sigma_70.pdf}
      \label{fig:Sigmacorr_sys_70}
    }\\

    \subfloat[][$\Sigma^{sys}_{corr}$ vs ($G_{calc}$ and $\Sigma_{calc}$) \\ for $N_{events}=80$  ] {
      \includegraphics[width=0.33\textwidth]{figures/Sys_corr/Graph_Sigma_80.pdf}
      \label{fig:Sigmacorr_sys_80}
    }
    \subfloat[][ $\Sigma^{sys}_{corr}$ vs ($G_{calc}$ and $\Sigma_{calc}$) \\ for $N_{events}=90$ ] {
      \includegraphics[width=0.33\textwidth]{figures/Sys_corr/Graph_Sigma_90.pdf}
      \label{fig:Sigmacorr_sys_90}
    }
    \subfloat[][ $\Sigma^{sys}_{corr}$ vs ($G_{calc}$ and $\Sigma_{calc}$) \\ for $N_{events}=100$ ] {
      \includegraphics[width=0.33\textwidth]{figures/Sys_corr/Graph_Sigma_100.pdf}
      \label{fig:Sigmacorr_sys_100}
    } \\
    \subfloat[][$\Sigma^{sys}_{corr}$ vs ($G_{calc}$ and $\Sigma_{calc}$) \\ for $N_{events}=110$ ] {
      \includegraphics[width=0.33\textwidth]{figures/Sys_corr/Graph_Sigma_110.pdf}
      \label{fig:Sigmacorr_sys_110}
    }
    \subfloat[][$\Sigma^{sys}_{corr}$ vs ($G_{calc}$ and $\Sigma_{calc}$) \\ for $N_{events}=120$ ] {
      \includegraphics[width=0.33\textwidth]{figures/Sys_corr/Graph_Sigma_120.pdf}
      \label{fig:Sigmacorr_sys_120}
    }     
    \subfloat[][$\Sigma^{sys}_{corr}$ vs ($G_{calc}$ and $\Sigma_{calc}$) \\ for $N_{events}=130$ ] {
      \includegraphics[width=0.33\textwidth]{figures/Sys_corr/Graph_Sigma_130.pdf}
      \label{fig:Sigmacorr_sys_130}
    } \\
   \subfloat[][$\Sigma^{sys}_{corr}$ vs ($G_{calc}$ and $\Sigma_{calc}$) \\ for $N_{events}=140$ ] {
      \includegraphics[width=0.33\textwidth]{figures/Sys_corr/Graph_Sigma_140.pdf}
      \label{fig:Sigmacorr_sys_140}
    }
    \subfloat[][$\Sigma^{sys}_{corr}$ vs ($G_{calc}$ and $\Sigma_{calc}$) \\ for $N_{events}=150$ ] {
      \includegraphics[width=0.33\textwidth]{figures/Sys_corr/Graph_Sigma_150.pdf}
      \label{fig:Sigmacorr_sys_150}
    }     
    \subfloat[][$\Sigma^{sys}_{corr}$ vs ($G_{calc}$ and $\Sigma_{calc}$) \\ for $N_{events}=160$ ] {
      \includegraphics[width=0.33\textwidth]{figures/Sys_corr/Graph_Sigma_160.pdf}
      \label{fig:Sigmacorr_sys_160}
    }
  \end{center}
\end{figure}
\begin{figure}[htb]
\ContinuedFloat
\begin{center}
    \subfloat[][$\Sigma^{sys}_{corr}$ vs ($G_{calc}$ and $\Sigma_{calc}$) \\ for $N_{events}=170$  ] {
      \includegraphics[width=0.33\textwidth]{figures/Sys_corr/Graph_Sigma_170.pdf}
      \label{fig:Sigmacorr_sys_170}
    }
    \subfloat[][ $\Sigma^{sys}_{corr}$ vs ($G_{calc}$ and $\Sigma_{calc}$) \\ for $N_{events}=180$ ] {
      \includegraphics[width=0.33\textwidth]{figures/Sys_corr/Graph_Sigma_180.pdf}
      \label{fig:Sigmacorr_sys_180}
    }
    \subfloat[][ $\Sigma^{sys}_{corr}$ vs ($G_{calc}$ and $\Sigma_{calc}$) \\ for $N_{events}=190$ ] {
      \includegraphics[width=0.33\textwidth]{figures/Sys_corr/Graph_Sigma_190.pdf}
      \label{fig:Sigmacorr_sys_190}
    } \\
    \subfloat[][$\Sigma^{sys}_{corr}$ vs ($G_{calc}$ and $\Sigma_{calc}$) \\ for $N_{events}=200$ ] {
      \includegraphics[width=0.33\textwidth]{figures/Sys_corr/Graph_Sigma_200.pdf}
      \label{fig:Sigmacorr_sys_200}
    }
    \subfloat[][$\Sigma^{sys}_{corr}$ vs ($G_{calc}$ and $\Sigma_{calc}$) \\ for $N_{events}=250$ ] {
      \includegraphics[width=0.33\textwidth]{figures/Sys_corr/Graph_Sigma_250.pdf}
      \label{fig:Sigmacorr_sys_250}
    }     
    \subfloat[][$\Sigma^{sys}_{corr}$ vs ($G_{calc}$ and $\Sigma_{calc}$) \\ for $N_{events}=300$ ] {
      \includegraphics[width=0.33\textwidth]{figures/Sys_corr/Graph_Sigma_300.pdf}
      \label{fig:Sigmacorr_sys_300}
    } \\
    \subfloat[][$\Sigma^{sys}_{corr}$ vs ($G_{calc}$ and $\Sigma_{calc}$) \\ for $N_{events}=350$ ] {
      \includegraphics[width=0.33\textwidth]{figures/Sys_corr/Graph_Sigma_350.pdf}
      \label{fig:Sigmacorr_sys_350}
    }
    \subfloat[][$\Sigma^{sys}_{corr}$ vs ($G_{calc}$ and $\Sigma_{calc}$) \\ for $N_{events}=400$ ] {
      \includegraphics[width=0.33\textwidth]{figures/Sys_corr/Graph_Sigma_400.pdf}
      \label{fig:Sigmacorr_sys_400}
    }     
    \subfloat[][$\Sigma^{sys}_{corr}$ vs ($G_{calc}$ and $\Sigma_{calc}$) \\ for $N_{events}=500$ ] {
      \includegraphics[width=0.33\textwidth]{figures/Sys_corr/Graph_Sigma_500.pdf}
      \label{fig:Sigmacorr_sys_500}
    }\\
    \subfloat[][$\Sigma^{sys}_{corr}$ vs ($G_{calc}$ and $\Sigma_{calc}$) \\ for $N_{events}=600$ ] {
      \includegraphics[width=0.33\textwidth]{figures/Sys_corr/Graph_Sigma_600.pdf}
      \label{fig:Sigmacorr_sys_600}
    }
    \subfloat[][$\Sigma^{sys}_{corr}$ vs ($G_{calc}$ and $\Sigma_{calc}$) \\ for $N_{events}=700$ ] {
      \includegraphics[width=0.33\textwidth]{figures/Sys_corr/Graph_Sigma_700.pdf}
      \label{fig:Sigmacorr_sys_700}
    }     
    \subfloat[][$\Sigma^{sys}_{corr}$ vs ($G_{calc}$ and $\Sigma_{calc}$) \\ for $N_{events}=800$ ] {
      \includegraphics[width=0.33\textwidth]{figures/Sys_corr/Graph_Sigma_800.pdf}
      \label{fig:Sigmacorr_sys_800}
    }\\
    \subfloat[][$\Sigma^{sys}_{corr}$ vs ($G_{calc}$ and $\Sigma_{calc}$) \\ for $N_{events}=900$ ] {
      \includegraphics[width=0.33\textwidth]{figures/Sys_corr/Graph_Sigma_200.pdf}
      \label{fig:Sigmacorr_sys_900}
    }
    \subfloat[][$\Sigma^{sys}_{corr}$ vs ($G_{calc}$ and $\Sigma_{calc}$) \\ for $N_{events}=1000$ ] {
      \includegraphics[width=0.33\textwidth]{figures/Sys_corr/Graph_Sigma_1000.pdf}
      \label{fig:Sigmacorr_sys_1000}
    }     
    \subfloat[][$\Sigma^{sys}_{corr}$ vs ($G_{calc}$ and $\Sigma_{calc}$) \\ for $N_{events}=2000$ ] {
      \includegraphics[width=0.33\textwidth]{figures/Sys_corr/Graph_Sigma_2000.pdf}
      \label{fig:Sigmacorr_sys_2000}
    }\\
    \subfloat[][$\Sigma^{sys}_{corr}$ vs ($G_{calc}$ and $\Sigma_{calc}$) \\ for $N_{events}=3000$ ] {
      \includegraphics[width=0.33\textwidth]{figures/Sys_corr/Graph_Sigma_3000.pdf}
      \label{fig:Sigmacorr_sys_3000}
    }
  \end{center}
\end{figure}

