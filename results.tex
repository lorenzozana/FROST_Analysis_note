\section{Results}
This section will present the experimental results for G and $\Sigma$ using the fit method described in section \ref{ch:extract_G}. 
Results are presented as a function of $E_{CMS}$ for different $cos(\theta)$ bins and vice-versa. 
In the following plots, as described in chapter \ref{ch:50pc_pol}, for each data point, in red is shown the possible fluctuation of varying the limit of 50\% for the photon polarization. 


\subsection{G vs \texorpdfstring{$E_{CMS}$}{E-CMS} for different \texorpdfstring{$cos(\theta)$}{cos(theta)} bins}\label{ch:result_W}
Different color bands have been used in order to show different coherent edge settings used in the data (summarized in Table \ref{table:g9a_conf_color}). One can see some overlapping in the photon energies (and therefore W) covered by different coherent edge settings. e.g.  around $E_{CMS} \sim 2 GeV$. One can check the consistency of the extraction of G from these overlapped measurements. Together with the data are plotted current MAID2007 \cite{MAID_2007} (blue line)  and SAID \cite{PhysRevC.86.015202} (red line) predictions, before this data is submitted.
\begin{table}[h]
  \begin{center}
    \begin{tabular}{ |l||l|l||l|l||l|l||l|l||}
      \hline
      \multicolumn{9}{|c|}{g9a Run period: Linearly polarized } \\
      \hline
      \multicolumn{3}{|l||}{$E_{beam}$} & \multicolumn{6}{|c||}{2.780 GeV}   \\
      \hline
      \multicolumn{3}{|l||}{$Coh_{Edge}$} & \multicolumn{2}{|c||}{0.73 GeV} &  \multicolumn{2}{|c||}{0.93 GeV} &  \multicolumn{2}{|c||}{1.1 GeV}   \\
      \hline
      \multicolumn{3}{|l||}{} & $E_{CMS}^{min} $ &  $E_{CMS}^{max} $ & $E_{CMS}^{min} $ &  $E_{CMS}^{max} $ & $E_{CMS}^{min} $ &  $E_{CMS}^{max} $\\
      \multicolumn{3}{|l||}{(GeV)} & 1.4 & 1.5 & 1.54 & 1.63 & 1.63 & 1.72  \\
      \hline
      \multicolumn{3}{|l||}{} & n  &  $\Delta E$ & n  &  $\Delta E$ & n  &  $\Delta E$ \\
      \multicolumn{3}{|l||}{(MeV)} & 4 & 25 & 3 & 30 & 3 & 30  \\
      \hline
      \hline
      $E_{beam}$&  \multicolumn{8}{|c||}{3.545 GeV} \\
      \hline
      $Coh_{Edge}$&  \multicolumn{2}{|c||}{1.1 GeV} &  \multicolumn{2}{|c||}{1.3 GeV} &  \multicolumn{2}{|c||}{1.5 GeV} &  \multicolumn{2}{|c||}{1.7 GeV} \\
      \hline
      & $E_{CMS}^{min} $ &  $E_{CMS}^{max} $& $E_{CMS}^{min} $ &  $E_{CMS}^{max} $ & $E_{CMS}^{min} $ &  $E_{CMS}^{max} $ & $E_{CMS}^{min} $ &  $E_{CMS}^{max} $ \\
      (GeV) & 1.63 & 1.72 & 1.71 & 1.83 & 1.835 & 1.925 &  1.93 & 2.02 \\
      \hline
      & n  &  $\Delta E$ & n  &  $\Delta E$ & n  &  $\Delta E$& n  &  $\Delta E$ \\
      (MeV) & 3 & 30 & 4 & 30 & 3 & 30 &3 &30 \\
      \hline
      \hline
      \multicolumn{3}{|l||}{$E_{beam}$} &  \multicolumn{6}{|c||}{4.599 GeV} \\
      \hline
      \multicolumn{3}{|l||}{$Coh_{Edge}$} &  \multicolumn{2}{|c||}{1.9 GeV} &  \multicolumn{2}{|c||}{2.1 GeV} &  \multicolumn{2}{|c||}{2.3 GeV}  \\
      \hline
      \multicolumn{3}{|l||}{} & $E_{CMS}^{min} $ &  $E_{CMS}^{max} $& $E_{CMS}^{min} $ &  $E_{CMS}^{max} $ & $E_{CMS}^{min} $ &  $E_{CMS}^{max} $ \\
      \multicolumn{3}{|l||}{(GeV)} & 1.945 & 2.105 & 2.05 & 2.2 & 2.19 & 2.29 \\
      \hline
      \multicolumn{3}{|l||}{} & n  &  $\Delta E$ & n  &  $\Delta E$ & n  &  $\Delta E$ \\
      \multicolumn{3}{|l||}{(MeV)} & 4 & 40 & 3 & 50 & 2 & 50  \\
      \hline
    \end{tabular}
  \end{center}
  \caption{g9a Run period: Different electron beam energies ($E_{beam}$) were used with different Energies for the Coherent Edge Polarization ($Coh_{Edge}$). The photon energy profile has a direct correlation with the Center of Mass Energy ($E_{CMS}$), which has a minimum and maximum limit chosen in order to reduce systematic errors in the evaluation of the beam polarization (with a Beam Polarization $>50\%$). The range used in each configuration ($E_{CMS}^{min} $ ,  $E_{CMS}^{max}$), the number of bins (n) and the size of each bin is also shown in this table.}
  \label{table:g9a_conf_color}
\end{table}

\begin{figure}[htb]
  \begin{center}
    \subfloat[][G vs W $ -1 <= cos(\theta_{CM})<-0.8$ (\color{blue}{MAID} \color{red}{SAID})] {
      \includegraphics[width=0.5\textwidth]{figures/G_plots/comp_system_0.png} 
      \label{fig:GvsW_theta1}
    }
    \subfloat[][G vs W $ -0.8 <= cos(\theta_{CM})<-0.6$ (\color{blue}{MAID} \color{red}{SAID})] {
      \includegraphics[width=0.5\textwidth]{figures/G_plots/comp_system_1.png}
    \label{fig:GvsW_theta2}
    } \\
    \subfloat[][G vs W $ -0.6 <= cos(\theta_{CM})<-0.4$ (\color{blue}{MAID} \color{red}{SAID})] {
    \includegraphics[width=0.5\textwidth]{figures/G_plots/comp_system_2.png}
    \label{fig:GvsW_theta3}
    }
    \subfloat[][G vs W $ -0.4 <= cos(\theta_{CM})<-0.2$ (\color{blue}{MAID} \color{red}{SAID})] {
      \includegraphics[width=0.5\textwidth]{figures/G_plots/comp_system_3.png}
      \label{fig:GvsW_theta4}
    }
  \end{center}
\end{figure}
\begin{figure}[htb]
\ContinuedFloat
  \begin{center} 
    \subfloat[][G vs W $ -0.2 <= cos(\theta_{CM})<0.0$ (\color{blue}{MAID} \color{red}{SAID})] {
      \includegraphics[width=0.5\textwidth]{figures/G_plots/comp_system_4.png}
      \label{fig:GvsW_theta5}
    }
    \subfloat[][G vs W $ 0.0 <= cos(\theta_{CM})<0.2$ (\color{blue}{MAID} \color{red}{SAID})] {
      \includegraphics[width=0.5\textwidth]{figures/G_plots/comp_system_5.png}
      \label{fig:GvsW_theta6}
    } \\    
    \subfloat[][G vs W $ 0.2 <= cos(\theta_{CM})<0.4$ (\color{blue}{MAID} \color{red}{SAID})] {
      \includegraphics[width=0.5\textwidth]{figures/G_plots/comp_system_6.png}
      \label{fig:GvsW_theta7}
    }
    \subfloat[][G vs W $ 0.4 <= cos(\theta_{CM})<0.6$ (\color{blue}{MAID} \color{red}{SAID})] {
      \includegraphics[width=0.5\textwidth]{figures/G_plots/comp_system_7.png}
      \label{fig:GvsW_theta8}
    }
  \end{center}
\end{figure}
\begin{figure}[htb]
\ContinuedFloat
  \begin{center} 
    \subfloat[][G vs W $ 0.6 <= cos(\theta_{CM})<0.8$ (\color{blue}{MAID} \color{red}{SAID})] {
      \includegraphics[width=0.5\textwidth]{figures/G_plots/comp_system_8.png}
      \label{fig:GvsW_theta9}
    }
    \subfloat[][G vs W $ 0.8 <= cos(\theta_{CM})<1.0$ (\color{blue}{MAID} \color{red}{SAID})] {
      \includegraphics[width=0.5\textwidth]{figures/G_plots/comp_system_9.png}
      \label{fig:GvsW_theta10}
    }
  \end{center}
\end{figure}

\FloatBarrier

\subsection{G vs  \texorpdfstring{$\theta_{CM}$}{theta-CM}  for different \texorpdfstring{$E_{CMS}$}{E-CMS} bins}\label{ch:result_th}
G is plotted for different $E_{CMS}$ as a function of $\theta_{CM}$. Bins have been chosen of equal size in $cos(\theta_{CM})$. 

\begin{figure}[htb]
  \begin{center}
    \subfloat[][G vs $\theta_{CM}$ ] {
      \includegraphics[width=0.33\textwidth]{figures/G_plots/theta_system_0.png}
      \label{fig:Gvstheta_W1}
    }
    \subfloat[][G vs $\theta_{CM}$ ] {
      \includegraphics[width=0.33\textwidth]{figures/G_plots/theta_system_1.png}
      \label{fig:Gvstheta_W2}
    }
    \subfloat[][G vs $\theta_{CM}$ ] {
      \includegraphics[width=0.33\textwidth]{figures/G_plots/theta_system_2.png}
      \label{fig:Gvstheta_W3}
    } \\
    \subfloat[][G vs $\theta_{CM}$ ] {
      \includegraphics[width=0.33\textwidth]{figures/G_plots/theta_system_3.png}
      \label{fig:Gvstheta_W4}
    }
    \subfloat[][G vs $\theta_{CM}$ ] {
      \includegraphics[width=0.33\textwidth]{figures/G_plots/theta_system_4.png}
      \label{fig:Gvstheta_W5}
    }
    \subfloat[][G vs $\theta_{CM}$ ] {
      \includegraphics[width=0.33\textwidth]{figures/G_plots/theta_system_5.png}
      \label{fig:Gvstheta_W6}
    } \\
    \subfloat[][G vs $\theta_{CM}$ ] {
      \includegraphics[width=0.33\textwidth]{figures/G_plots/theta_system_6.png}
      \label{fig:Gvstheta_W7}
    }
    \subfloat[][G vs $\theta_{CM}$ ] {
      \includegraphics[width=0.33\textwidth]{figures/G_plots/theta_system_7.png}
      \label{fig:Gvstheta_W8}
    }
    \subfloat[][G vs $\theta_{CM}$ ] {
      \includegraphics[width=0.33\textwidth]{figures/G_plots/theta_system_8.png}
      \label{fig:Gvstheta_W9}
    } \\
  \end{center}
\end{figure}
\begin{figure}[htb]
\ContinuedFloat
  \begin{center} 
    \subfloat[][G vs $\theta_{CM}$ ] {
      \includegraphics[width=0.33\textwidth]{figures/G_plots/theta_system_9.png}
      \label{fig:Gvstheta_W10}
    }
    \subfloat[][G vs $\theta_{CM}$ ] {
      \includegraphics[width=0.33\textwidth]{figures/G_plots/theta_system_10.png}
      \label{fig:Gvstheta_W11}
    }
    \subfloat[][G vs $\theta_{CM}$ ] {
      \includegraphics[width=0.33\textwidth]{figures/G_plots/theta_system_11.png}
      \label{fig:Gvstheta_W12}
    } \\
    \subfloat[][G vs $\theta_{CM}$ ] {
      \includegraphics[width=0.33\textwidth]{figures/G_plots/theta_system_12.png}
      \label{fig:Gvstheta_W13}
    }
    \subfloat[][G vs $\theta_{CM}$ ] {
      \includegraphics[width=0.33\textwidth]{figures/G_plots/theta_system_13.png}
      \label{fig:Gvstheta_W14}
    }
    \subfloat[][G vs $\theta_{CM}$ ] {
      \includegraphics[width=0.33\textwidth]{figures/G_plots/theta_system_14.png}
      \label{fig:Gvstheta_W15}
    } \\
    \subfloat[][G vs $\theta_{CM}$ ] {
      \includegraphics[width=0.33\textwidth]{figures/G_plots/theta_system_15.png}
      \label{fig:Gvstheta_W16}
    }
    \subfloat[][G vs $\theta_{CM}$ ] {
      \includegraphics[width=0.33\textwidth]{figures/G_plots/theta_system_16.png}
      \label{fig:Gvstheta_W17}
    }
    \subfloat[][G vs $\theta_{CM}$ ] {
      \includegraphics[width=0.33\textwidth]{figures/G_plots/theta_system_17.png}
      \label{fig:Gvstheta_W18}
    } \\
    \subfloat[][G vs $\theta_{CM}$ ] {
      \includegraphics[width=0.33\textwidth]{figures/G_plots/theta_system_18.png}
      \label{fig:Gvstheta_W19}
    }
    \subfloat[][G vs $\theta_{CM}$ ] {
      \includegraphics[width=0.33\textwidth]{figures/G_plots/theta_system_19.png}
      \label{fig:Gvstheta_W20}
    }
    \subfloat[][G vs $\theta_{CM}$ ] {
      \includegraphics[width=0.33\textwidth]{figures/G_plots/theta_system_20.png}
      \label{fig:Gvstheta_W21}
    }
  \end{center}
\end{figure}
\begin{figure}[htb]
\ContinuedFloat
  \begin{center} 
    \subfloat[][G vs $\theta_{CM}$ ] {
      \includegraphics[width=0.33\textwidth]{figures/G_plots/theta_system_21.png}
      \label{fig:Gvstheta_W22}
    }
    \subfloat[][G vs $\theta_{CM}$ ] {
      \includegraphics[width=0.33\textwidth]{figures/G_plots/theta_system_22.png}
      \label{fig:Gvstheta_W23}
    }
    \subfloat[][G vs $\theta_{CM}$ ] {
      \includegraphics[width=0.33\textwidth]{figures/G_plots/theta_system_23.png}
      \label{fig:Gvstheta_W24}
    } \\
    \subfloat[][G vs $\theta_{CM}$ ] {
      \includegraphics[width=0.33\textwidth]{figures/G_plots/theta_system_24.png}
      \label{fig:Gvstheta_W25}
    }
    \subfloat[][G vs $\theta_{CM}$ ] {
      \includegraphics[width=0.33\textwidth]{figures/G_plots/theta_system_25.png}
      \label{fig:Gvstheta_W26}
    }
    \subfloat[][G vs $\theta_{CM}$ ] {
      \includegraphics[width=0.33\textwidth]{figures/G_plots/theta_system_26.png}
      \label{fig:Gvstheta_W27}
    } \\
    \subfloat[][G vs $\theta_{CM}$ ] {
      \includegraphics[width=0.33\textwidth]{figures/G_plots/theta_system_27.png}
      \label{fig:Gvstheta_W28}
    }
    \subfloat[][G vs $\theta_{CM}$ ] {
      \includegraphics[width=0.33\textwidth]{figures/G_plots/theta_system_28.png}
      \label{fig:Gvstheta_W29}
    }
  \end{center}
\end{figure}




\subsection{Discussion}

The g9a FROST experiment has enabled G and $\Sigma$ to be extracts with good statistical accuracy in the range W=1.4GeV to 2.29GeV and $cos(\theta_{CM})$ -1 to 1.  This is the first measuremenmt of G over a wide W and theta range for the $\pi^+ N$ photo-production channel. The data will provide valuable new constraints on partial wave analyses to constrain the N* spectrum. Distributing the data to the PWA groups and investigating the effect on resonance properties will be the next step once the data analysis has been approved. 
